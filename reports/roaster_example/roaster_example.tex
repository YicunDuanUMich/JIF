%
%  roaster_example.tex
%
%  Created by Michael D. Schneider on 2015-11-02
%    <schneider@ucdavis.edu>
%
\documentclass[11pt, letterpaper]{article}

\usepackage[OT1]{fontenc}
\usepackage[in]{fullpage}

\usepackage[pdftex,pdfpagemode={UseOutlines},bookmarks,bookmarksopen,colorlinks,linkcolor={black},citecolor={black},urlcolor={red}]{hyperref}
\usepackage{graphicx}
\usepackage{bm}
\usepackage{amsmath,amssymb}
\usepackage{aas_macros}

\usepackage{mathptmx}  % times roman, including math (where possible)

% Commands for colorful collaborative markup
% (copied from http://www.astrobetter.com/whats-the-best-tool-to-annotate-pdfs/#comment-10303)
% \usepackage[usenames]{color}
% \newcommand{\mike}[1]{\textcolor{Red}{\bf #1}}
% \newcommand{\collaborator}[1]{\textcolor{ForestGreen}{\bf #1}}

% \setlength{\parskip}{6pt}

% \renewcommand*\contentsname{Agenda}
\setlength{\headsep}{5pt}

% Macros
\newcommand{\half}{\frac{1}{2}}
\newcommand{\rhocrit}{\rho_{\rm crit}}
\newcommand{\rvir}{r_{\rm vir}}
\newcommand{\mvir}{m_{\rm vir}}
\newcommand{\om}{\Omega_{m}}
\newcommand{\kv}{\mathbf{k}}
\newcommand{\xv}{\mathbf{x}}
\newcommand{\hmsun}{h^{-1}M_{\odot}}
\newcommand{\hmpc}{h^{-1}Mpc}
\newcommand{\hgpc}{h^{-1}{\rm Gpc}}


\usepackage{fancyhdr}
\pagestyle{fancy}
\lhead{JIF: Roaster example}
\rhead{\today}
\cfoot{\input{git_tag.tex}}
\rfoot{\thepage}
\renewcommand{\headrulewidth}{0.4pt}
\renewcommand{\footrulewidth}{0.4pt}

%%%%%%%%%%%%%%%%%%%%%%%%%%%%%%%%%%%%%%%%%%%%%%%%%%%%%%%%%%%%%%%%%%%%%%%%%%%%%%%
\begin{document}

% \title{\vspace{-4ex}JIF: Roaster example\vspace{-4ex}}

% \author{Michael D. Schneider}
%$^{1} Livermore National Laboratory, P.O. Box 808 L-210, Livermore, CA 94551-0808, USA.

% \date{\today}

% \maketitle

\begin{figure}
  \centerline{
    \includegraphics[width=0.5\textwidth]{../../TestData/test_lsst_image.png}
    \includegraphics[width=0.5\textwidth]{../../TestData/test_wfirst_image.png}
  }
  \centerline{
    \includegraphics[width=0.5\textwidth]{../../TestData/test_lsst_psf.png}
    \includegraphics[width=0.5\textwidth]{../../TestData/test_wfirst_psf.png}
  }
  \caption{Input data generated by \texttt{galsim\_galaxy.py}.}
  \label{fig:input_data}
\end{figure}


\begin{figure}
  \centerline{
    \includegraphics[width=\textwidth]{../../output/roasting/roaster_inspector_walkers.png}
  }
  \caption{Trace of the MCMC steps from \texttt{Roaster.py}. Colors indicate
  different \texttt{emcee} walkers.}
\end{figure}


\begin{figure}
  \centerline{
    \includegraphics[width=\textwidth]{../../output/roasting/roaster_inspector_triangle.png}
  }
  \caption{Marginal posteriors from \texttt{Roaster.py}. Vertical lines indicate
  `truth' values.}
\end{figure}

\begin{figure}
  \centerline{
    \includegraphics[width=\textwidth]{../../output/roasting/roaster_data_and_model_0.png}
  }
  \caption{Pixel data, model, and residuals for LSST.}
\end{figure}

\begin{figure}
  \centerline{
    \includegraphics[width=\textwidth]{../../output/roasting/roaster_data_and_model_1.png}
  }
  \caption{Pixel data, model, and residuals for WFIRST.}
\end{figure}

\end{document}
